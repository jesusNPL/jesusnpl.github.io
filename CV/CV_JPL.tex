%%%%%%%%%%%%%%%%%%%%%%%%%%%%%%%%%%%%%%%%%
% This document is based on a template available at
% http://www.LaTeXTemplates.com
%
% Original author:
% Trey Hunner (http://www.treyhunner.com/)
%%%%%%%%%%%%%%%%%%%%%%%%%%%%%%%%%%%%%%%%%

%----------------------------------------------------------------------------------------
%	PACKAGES AND OTHER DOCUMENT CONFIGURATIONS
%----------------------------------------------------------------------------------------

\documentclass{resume} % Use the custom resume.cls style

\usepackage[left=0.75in,top=0.6in,right=0.75in,bottom=0.6in]{geometry} % Document margins
\usepackage{fancyhdr}
\usepackage{lastpage}
\usepackage{amssymb}
\usepackage{url}
\usepackage{multicol}
%\usepackage{amssymb}
%\usepackage{amsmath}
\usepackage[euler]{textgreek}
%\usepackage{libertine}
%\usepackage[T1]{fontenc} 
%\input{officialheader.tex} %If you want this file to compile to my CV, comment out this line and uncomment the next three
\name{Jesús N. Pinto-Ledezma} 
\address{\url{jpintole@umn.edu} \\ \url{https://jesusnpl.github.io}  \\ @JesusNPL } 
\address{Department of Ecology, Evolution and Behavior \\ University of Minnesota \\ St Paul, MN 55108} 

\pagestyle{fancy}
\fancyhf{}
\renewcommand{\headrulewidth}{0pt}
\cfoot{Pinto-Ledezma, page \thepage\ of \pageref{LastPage}}

\begin{document}

\begin{rSection}{Research Interests}

{\normalfont I am an evolutionary and quantitative ecologist whose work focuses on developing a deeper understanding of species coexistence and patterns of diversity across spatial and temporal scales, and the underlying processes that drive, maintain and alter such patterns. I have a passion for science and for diversity and inclusion in education and research.}

\end{rSection}

%----------------------------------------------------------------------------------------
%	EDUCATION
%----------------------------------------------------------------------------------------

\begin{rSection}{Education}
{\bf Ph.D., Ecology and Evolution} \hfill \normalfont 2013-2017 \\ 
{\bf Universidade Federal de Goiás - Goiânia} \hfill \normalfont Goiás, Brazil \\
{\normalfont Dissertation: \emph{Origin and assembly of Furnariides assemblages across space and time: the role of historical processes}} \\
{\normalfont Advisor: José Alexandre Felizola Diniz-Filho} \smallskip 

{\bf M.S., Wildlife Management} \hfill \normalfont 2006-2009 \\
{\bf Universidad Nacional de Córdoba - Córdoba} \hfill \normalfont Córdoba, Argentina \\
{\normalfont Dissertation: \emph{Determination of special protected areas for the conservation of migratory birds in the Mar Chiquita Reserve}} \\
{\normalfont Advisors: Adrian H. Farmer and Enrique H. Bucher} \smallskip 

{\bf B.A., Biology} \hfill \normalfont 2001-2005 \\
{\bf Universidad Autónoma Gabriel René Moreno} \hfill \normalfont Santa Cruz, Bolivia \\
{\normalfont Distinction in All Subjects. \emph{Cum Laude Honors}} \\ 
{\normalfont Advisor: Teresa Ruiz de Centurión} \smallskip 
\end{rSection}

%----------------------------------------------------------------------------------------
%	ACADEMIC APPOINTMENTS
%----------------------------------------------------------------------------------------

\begin{rSection}{Professional Appointments}

{\bf UMN Presidential Fellow} \hfill {\normalfont 2022-present} \\
{\normalfont University of Minnesota, Department of Ecology Evolution and Behavior}\hfill {\normalfont St Paul, MN, USA}

{\bf Research Scientist} \hfill {\normalfont 2020-2022} \\
{\normalfont University of Minnesota, Department of Ecology Evolution and Behavior}\hfill {\normalfont St Paul, MN, USA} 

{\bf Grand Challenge in Biology Postdoctoral Fellow} \hfill {\normalfont 2017-2020} \\
{\normalfont University of Minnesota, Department of Ecology Evolution and Behavior}\hfill {\normalfont St Paul, MN, USA} 

{\bf Research Associate} \hfill {\normalfont 2009-present} \\
{\normalfont Museo de Historia Natural Noel Kempff Mercado}\hfill {\normalfont Santa Cruz, Bolivia} \\
\emph{\normalfont Ad Honorem} 

{\bf Guest Lecturer} \hfill {\normalfont 2012-2013} \\
{\normalfont Carrera de Biología, Universidad Autónoma Gabriel René Moreno}\hfill {\normalfont Santa Cruz, Bolivia} 

{\bf Visiting Researcher} \hfill {\normalfont 2010-2011} \\
{\normalfont Centro de Pesquisas do Pantanal, Universidade Federal de Mato Grosso}\hfill {\normalfont Cuiabá, Bolivia} 

{\bf Intern} \hfill {\normalfont 2003-2006} \\
{\normalfont Museo de Historia Natural Noel Kempff Mercado}\hfill {\normalfont Santa Cruz, Brazil} 

{\bf Bolivian Military Service} \hfill {\normalfont 2000-2001} \\
{\normalfont Air Force}\hfill {\normalfont Santa Cruz, Bolivia} 
\end{rSection}

%\clearpage

%----------------------------------------------------------------------------------------
%	AWARDS AND FELLOWSHIPS
%----------------------------------------------------------------------------------------

\begin{rSection}{Awards and Fellowships}

\begin{esSubsection}{President's Postdoctoral Fellowship Program, }{\normalfont University of Minnesota, College of Biological Sciences }{\normalfont 2022-present}{}{}
\end{esSubsection}

\begin{esSubsection}{AAAS/Science Membership Award, }{\normalfont the American Association for the Advancement of Science Program for Excellence in Science}{\normalfont 2020-present}{}{}
\end{esSubsection}

\begin{esSubsection}{Sigma Xi, Membership, }{ \textsc{Sigma Xi} (\textSigma \textXi) \normalfont the Scientific Research Honor Society}{\normalfont 2021}{}{}
\end{esSubsection}

\begin{esSubsection}{Grand Challenges in Biology Postdoctoral Program, }{\normalfont University of Minnesota, College of Biological Sciences }{\normalfont 2017-2020}{}{}
\end{esSubsection}

\begin{esSubsection}{CAPES PhD fellowship, }{\normalfont Coordination for the Improvement of Higher Education Personnel, Brazil }{\normalfont 2015-2017}{}{}
\end{esSubsection}

\begin{esSubsection}{OEA-CGUB Doctoral Scholarship, }{\normalfont Organization of American States (OAS) and the Coimbra Group of Brazilian Universities (GCUB), Brazil }{\normalfont 2014-2015}{}{}
\end{esSubsection}

\begin{esSubsection}{Master’s Program in Wildlife Management, }{\normalfont US Wildlife Service, Universidad Nacional de Córdoba, Córdoba, Argetina }{\normalfont 2006-2008}{}{}
\end{esSubsection}

\begin{esSubsection}{ISSLR Membership and Travel Award, }{\normalfont International Society of Salt Lake Research }{\normalfont 2011}{}{}
\end{esSubsection}

\begin{esSubsection}{SWS Membership Award, }{\normalfont Society of Wetlands Scientists }{\normalfont 2010-2013}{}{}
\end{esSubsection}

\begin{esSubsection}{SCB Membership Award, }{\normalfont Society for Conservation Biology - A global community of conservation professionals }{\normalfont 2007-2009}{}{}
\end{esSubsection}

\begin{esSubsection}{Best Student Award for the Biology Major, }{\normalfont Universidad Autónoma Gabriel René Moreno, Santa Cruz de la Sierra, Bolivia}{\normalfont 2005}{}{}
\end{esSubsection}{}
\end{rSection}


%----------------------------------------------------------------------------------------
%	RESEARCH ACTIVITIES
%----------------------------------------------------------------------------------------

\begin{rSection}{Research Activities}
%\textbf{PENDING FEDERAL GRANTS:}
\textbf{FEDERAL GRANTS:} \hfill {\em Total awarded USD 5,886,664}

\begin{pSubsection}{NASA ROSES Biodiversity: }{\normalfont Mapping changes in forest diversity and disease in North American temperate forests.}{\normalfont 2021-2024}{\normalfont Role: Co-Investigator}{\normalfont Cavender-Bares, Jeannine (PI, UMN), Townsend, Philip (co-PI, UW). | {\bf Award: USD 481,933.}}
\end{pSubsection}

\begin{pSubsection}{National Science Foundation, MSA: }{\normalfont Integrating biodiversity observations with airborne and satellite data to predict shifts in assemblage diversity and composition under global change.}{\normalfont 2020-2023}{\normalfont Role: Principal Investigator}{}{{\bf Pinto-Ledezma, Jesús N.} \normalfont (PI, UMN), Cavender-Bares, Jeannine (co-PI, UMN). | {\bf Award: USD 299,375.}}
\end{pSubsection}

\begin{pSubsection}{National Science Foundation, BII Implementation: }{\normalfont The causes and consequences of plant biodiversity across scales in a rapidly changing world.}{\normalfont 2020-2025}{\normalfont Role: Co-Investigator}{\normalfont Cavender-Bares, Jeannine (PI, UMN), Townsend, Philip (co-PI, UW), Reich, Peter (co-PI, UMN), José E. Meireles (co-PI, UMaine), Amy Trowbridge (co-PI, UW). More information at: \url{https://www.spectralbiology.org}. | {\bf Total award: USD 12,5000,000.} Awarded to date: USD 5,105,356.}
\end{pSubsection}

\textbf{NON-FEDERAL GRANTS:}  \hfill {\em Total awarded USD 195,508}

\begin{pSubsection}{Cedar Creek Ecological Synthesis Working Groups: }{\normalfont Evolutionary imprints on the species responses to the varying environment.}{\normalfont 2023-2024}{\normalfont Role: Principal Investigator}{Pinto-Ledezma, Jesús N. \normalfont (PI, UMN), Cavender-Bares, Jeannine (co-PI, UMN), Borer, Elizabeth (co-PI, UMN).}{ | {\bf Award: USD 7,000.}}
\end{pSubsection}

\begin{pSubsection}{University of Minnesota, President's Postdoctoral Fellowship Program: }{\normalfont Dispersal as a bridge between ecology, evolution, and behavior.}{\normalfont 2022-2024}{\normalfont Role: Presidential Fellow}{ | {\bf Award: USD 162,281.}}
\end{pSubsection}

\begin{pSubsection}{College of Biological Sciences, UMN, Grand Challenges in Biology Postdoctoral Fellowship: }{\normalfont Evaluating the roles of ecological and historical processes in biological invasions.}{\normalfont 2017-2020}{\normalfont Role: Postdoctoral Fellow}{ | {\bf Award: USD 157,500.}}
\end{pSubsection}

\begin{pSubsection}{Academia Nacional de Ciencias de Bolivia, Capitulo Santa Cruz: }{\normalfont Amphibians as a model of biological control in agricultural areas of central Santa Cruz, Bolivia.}{2016-2017}{\normalfont Role: Co-Principal Investigator}{\normalfont Pinto, Marco Aurelio (PI), \textbf{Pinto-Ledezma, Jesús N.} (Co-PI). | {\bf Award: USD 1,500.}}
\end{pSubsection}

\begin{pSubsection}{Rufford Foundation: }{\normalfont Rescuing the biodiversity of the Cerro Mutún: a basis for generation the conservation measures for Bolivian biodiversity.}{\normalfont 2016-2017}{\normalfont Role: Co-Principal Investigator}{\normalfont Villarroel, Daniel (PI), \textbf{Pinto-Ledezma, Jesus N.} (Co-PI). | {\bf Award: USD 7,674.}}
\end{pSubsection}

\begin{pSubsection}{Rufford Foundation: }{\normalfont Long-Term Effects of Habitat Modification on Amphibians in the Yungas and Inter-Andean Dry Valley Ecoregions.}{\normalfont 2013-2014}{\normalfont Role: Co-Principal Investigator}{\normalfont Sosa, Ronald (PI), \textbf{Pinto-Ledezma, Jesús N.} (Co-PI). | {\bf Award: USD 6,568.}}
\end{pSubsection}

\begin{pSubsection}{Rufford Foundation: }{\normalfont The Hyacinth Macaw Program: Population Status and Conservation of the Hyacinth Macaw.}{\normalfont 2013-2014}{\normalfont Role: Principal Investigator}{\textbf{Pinto-Ledezma, Jesús N.} \normalfont (PI). | {\bf Award: USD 7,168.}}
\end{pSubsection}

\begin{pSubsection}{Academia Nacional de Ciencias de Bolivia, Capítulo Santa Cruz: }{\normalfont Analysis of effect of the land use change on amphibian communities in the Mutun region.}{\normalfont 2012-2013}{\normalfont Role: Principal Investigator}{\textbf{Pinto-Ledezma, Jesús N.} \normalfont (PI). | {\bf Award: USD 1,500.}}
\end{pSubsection}

\begin{pSubsection}{Academia Nacional de Ciencias de Bolivia, Capítulo Santa Cruz: }{\normalfont Areas for the conservation of the Hyacinth macaw.}{\normalfont 2011-2012}{\normalfont Role: Principal Investigator}{\textbf{Pinto-Ledezma, Jesús N.} \normalfont (PI). | {\bf Award: USD 1,500.}}
\end{pSubsection}

\begin{pSubsection}{Rufford Foundation: }{\normalfont Testing a Habitat Model for the Hyacinth macaw (\emph{Anodorhynchus hyacinthinus}) and Mapping HS for the Species in Protected Areas in Bolivian Pantanal.}{\normalfont 2009-2011}{\normalfont Role: Principal Investigator}{\textbf{Pinto-Ledezma, Jesús N.} \normalfont (PI). | {\bf Award: USD 5,098.}}
\end{pSubsection}

\end{rSection}

%----------------------------------------------------------------------------------------
%	PUBLICATIONS
%----------------------------------------------------------------------------------------

\begin{rSection}{Publications}

\normalfont As of July 2023, I have published 31 articles in indexed journals, 12 articles in other peer-reviewed journals (non-indexed journals), 4 peer-reviewed book chapters and 4 non peer-reviewed book chapters.

\dag {\em Equal contribution} \\
{\em *Undergraduate student}

{31.} {\bf{J.N. Pinto-Ledezma}}, {S. Díaz, B.S. Halpern, C. Khoury and J. Cavender-Bares. (Accepted). No branch left behind: tracking terrestrial biodiversity from a phylogenetic completeness perspective.} {\em Frontiers in Ecology and the Environment.} 

{30.} {Moulatlet, G.M., B. Kusumoto,} {\bf{J.N. Pinto-Ledezma}}, { T. Shiono, Y. Kubota and F. Villalobos. (Accepted). Global patterns of phylogenetic beta-diversity components in angiosperms.} {\em Journal of Vegetation Science.}

{29.} {Pellegrini, A.F., L. Anderegg, } {\bf{J.N. Pinto-Ledezma}}, {J. Cavender-Bares, S.E. Hobbie and P.B. Reich. ({\bf{2023}}). Consistent physiological, ecological, and evolutionary effects of fire regime on conservative leaf economics strategies plant communities.} {\em Ecology Letters, 26(4): 597–608.}

{28.} {Meltesen, K.M., E.T. Whiting,} {\bf{J.N. Pinto-Ledezma}}, {T.S. Cicak and D.L. Fox. ({\bf{2023}}). Deconstructing the latitudinal diversity gradient of North American mammals by nominal order.} {\em Journal of Mammalogy, gyad042.}

{27.} {Souza, K., D. Fortunato, L. Jardim, L.C. Terribile, M. Lima-Ribeiro, C. Mariano,} {\bf{J.N. Pinto-Ledezma}}, {L. M. Bini, R. Loyola, R. Dobrovolski, T.F.L.V.B. Rangel, I.F. Machado, T. Rocha, M.C. Batista, M.L. Lorini, M.M. Vale, C.A. Navas, N.M. Maciel, F. Villalobos, M.A. Olalla-Tarraga, J.F.M. Rodrigues, S. Gouveia and J.A.F. Diniz-Filho. ({\bf{2023}}). Evolutionary rescue and geographic range shifts under climate change for global amphibians.} {\em Frontiers in Ecology and Evolution, 11: 1038018.}

{26.} {Rueda-Cediel, P., R. Brain, N. Galic, } {\bf{J.N. Pinto-Ledezma}}, {A. Rico and V. Forbes. ({\bf{2023}}). Using life-history trait variation to inform ecological risk assessments for threatened and endangered plant species.} {\em Integrated Environmental Assessment and Management, 19(10): 213–223.}

{25.} {Velasco\dag, J.A. and} {\bf{J.N. Pinto-Ledezma\dag}}. {({\bf{2022}}). Mapping diversification metrics in macroecological studies: prospects and challenges.} {\em Frontiers in Ecology and Evolution, 10: 951271.}

{24.}  {Vargas, G., N. Kunert, W.M. Hammond, Z.C. Berry, L.K. Werden, C.M. Smith-Martin, B.T. Wolfe, L. Toro, A. Mondragón-Botero, } {\bf{J.N. Pinto-Ledezma}}, {N.B. Schwartz, M. Uriarte, L. Sack, K.J. Anderson-Teixeira and J.S. Powers. ({\bf{2022}}). Leaf habit affects the distribution of drought sensitivity but not water transport efficiency in the tropics.} {\em Ecology Letters, 25(12): 2637–2650.}

{23.} {Arango, A.,} {\bf{J.N. Pinto-Ledezma}}, {O. Soto-Rojas, A.M. Lindsay, Ch.D. Mendenhall and F. Villalobos. ({\bf{2022}}). Hand-Wing Index as a surrogate for dispersal: the case of the Emberizoidea radiation.} {\em Biological Journal of the Linnean Society, 137(1): 137–144.}

{22.} {Fontes, C.,} {\bf{J.N. Pinto-Ledezma}}, {A.L. Jacobsen, R.B. Pratt and J. Cavender-Bares. ({\bf{2022}}). Adaptive variation among oaks in wood anatomical properties is shaped by climate of origin and shows limited plasticity across environments.} {\em Functional Ecology, 36(2): 326-340.}

{21.} {Chaplin-Kramer, R., K.A. Brauman, J. Cavender-Bares, S. Díaz, G.T. Duarte, B.J. Enquist, L.A. Garibaldi, J. Geldmann, B.S. Halpern, T.W. Hertel, C.K. Khoury, J.M. Krieger, S. Lavorel, T. Mueller, R.A. Neugarten,} {\bf{J.N. Pinto-Ledezma}}, {S. Polasky, A. Purvis, V. Reyes-García, P.R. Roehrdanz, L.J. Shannon, M.R. Shaw, B.N. Strassburg, J.M. Tylianakis, P.H. Verburg, P. Visconti and N. Zafra-Calvo. ({\bf{2022}}). Conservation needs to integrate knowledge across scales.} {\em Nature Ecology and Evolution, 6: 118-119.}

{20.} {\bf{Pinto-Ledezma, J.N.}} {and J. Cavender-Bares. ({{\bf 2021}}). Predicting species distributions and community composition using satellite remote sensing predictors.} {\em Scientific Reports, 11: 16448.}

{19.} {Cavender-Bares, J., P. Reich, P.A. Townsend, A. Banerjee, E. Butler, A. Desai, A. Gevens, S. Hobbie, F. Isbell, E. Laliberté, J.E. Meireles, H. Menninger, R.P. Pavlick, {\bf{J.N. Pinto-Ledezma}}, C. Potter, M.C. Schuman, N. Springer, A. Stefanski, P. Trivedi, A. Trowbridge, L. Williams, C.G. Willis and Y. Yang. ({{\bf 2021}}). BII-Implementation: The causes and consequences of plant biodiversity across scales in a rapidly changing world.} {\em Research Ideas and Outcomes, 7: e63850}. 

{18.} {\bf{Pinto-Ledezma, J.N.}}, {F. Villalobos, P. Reich, J. Catford, D. Larkin and J. Cavender-Bares. ({{\bf 2020}}). Testing Darwin's naturalization conundrum based on taxonomic, phylogenetic and functional dimensions of vascular plant diversity.} {\em Ecological Monographs, 90(4): e01420}.

{17.} {Cavender-Bares, J., C. Fontes and} {\bf{J.N. Pinto-Ledezma}}. {({{\bf 2020}}). Open questions in understanding the adaptive significance of plant functional trait variation within a single lineage.} {\em New Phytologist, 227(3): 659-663}.

{16.} {\bf{Pinto-Ledezma, J.N.}}, {A.E. Jahn, V.R. Cueto, J.A.F. Diniz-Filho and F. Villalobos. ({{\bf 2019}}). Drives of phylogenetic assemblage structure of the Furnariides, a widespread clade of lowland Neotropical birds.} {\em The American Naturalist, 193(2): E41-E56}. 

{15.} {\bf{Pinto-Ledezma, J.N.}}, {D. Larkin and J. Cavender-Bares. ({{\bf 2018}}). Patterns of beta diversity of vascular plants and their correspondence with biome boundaries across North America.} {\em Frontiers in Ecology and Evolution, 6: 194}.

{14.} {Pereira, E.,} {\bf{J.N. Pinto-Ledezma}}, {C. de Freitas, F. Villalobos, R. Collevati and N. Medeiros. ({{\bf 2017}}). Evolution of anuran foam nest: trait conservatism and lineage diversification.} {\em Biological Journal of the Linnean Society 122(4): 814-823}. 

{13.} {\bf{Pinto-Ledezma, J.N.}}, {L. Simon, J.A.F Diniz-Filho and F. Villalobos. ({{\bf 2017}}). The geographic diversification of Furnariides: the role of forest versus open habitats in driving species richness gradients.} {\em Journal of Biogeography, 44(8): 1683-1693}. 

{12.} {Cseko, E., W. Franca-Rocha, T. Moura and} {\bf{J.N. Pinto-Ledezma}}. {({{\bf 2017}}). New range limit of the Broad-tipped Hermit ({\em Anopetia gounellei}, Aves: Trochilidae): the state of art and a review on the range area.} {\em Pápeis avulsos de Zoologia, 57(21): 275-285}. 

{11.} {*Sosa R., Ch. Schalk, L. Braga and} {\bf{J.N. Pinto-Ledezma}}. {({{\bf 2015}}). Geographic Distribution: {\em Rhinella amboroensis} (Cochabamba toad)} {\em Herpetological Review, 46(2): 214}.

{10.} {\bf{Pinto-Ledezma, J.N.}} {and *M.L. Rivero-Mamani. ({{\bf 2014}}). Temporal patterns of deforestation and fragmentation in Lowland Bolivia: Implications for climate change.} {\em Climatic Change, 127: 43-54}. 

{9.} {\bf{Pinto-Ledezma, J.N.}}, {V.X. Sandoval, V.N. Pérez, T.J. Caballero, *K. Mano, *M.A. Pinto and *R. Sosa. ({{\bf 2014}}). A spatial explicit habitat model for the Hyacinth Macaw ({\em Anodorhynchus hyacinthinus}) in the Bolivian Pantanal (Santa Cruz, Bolivia).} {\em Ecología en Bolivia, 49(2): 1605-2528}.

{8.} {Jahn A.E., D.J. Levey, V. Cueto,} {\bf{J.N. Pinto-Ledezma}}, {D. Tuero, J.W. Fox and D. Masson. ({{\bf 2013}}). Patterns of long-distance bird migration in South America as revealed by light-level geolocators.} {\em The Auk, 130(2): 223-229}. 

{7.} {Jahn A.E., V. Cueto, J.W. Fox, M.S. Husak,} {\bf{J.N. Pinto-Ledezma}}, {D.H. Kim, D.V. Landoll, H.K. Lepage, D.J. Levey, M.T. Murphy and R.B. Renfrew. ({{\bf 2013}}) Migration timing and wintering areas of three species of Tyrannus flycatchers breeding in the great plains of North America} {\em The Auk, 130(2): 247-257}. 

{6.} {*Sosa R., Ch. Schalk, L. Braga and} {\bf{J.N. Pinto-Ledezma}}. {({{\bf 2013}}). {\em Micrurus serranus} (NCN) diet.} {\em Herpetological Review, 44(1): 155}. 

{5.} {*Sosa R., Ch. Schalk, L. Braga and} {\bf{J.N. Pinto-Ledezma}}. {({{\bf 2013}}). {\em Phylodryas psammohidea} (Gunther's green racer) diet.} {\em Herpetological Bulletin, 124: 24}. 

{4.} {*Sosa R., Ch. Schalk, L. Braga and} {\bf{J.N. Pinto-Ledezma}}. {({{\bf 2012}}). {\em Clelia langeri} (NCN) diet.} {\em Herpetological Review, 43(4): 657}. 

{3.} {Jahn A.E.,} {\bf{J.N. Pinto-Ledezma}}, {A.M. Mamani, L.W. De Groote and D.J. Levey. ({{\bf 2010}}). Patterns of home range size and habitat occupancy of Tropical Kingbird ({\em Tyrannus m. melancholicus}) in the southern Amazon Basin.} {\em Ornitología Neotropical, 12: 39-46}. 

{2.} {\bf{Pinto-Ledezma, J.N.}} {and T. Ruiz de Centurión. ({{\bf 2010}}). Deforestation and fragmentation patterns 1976-2006 in San Julian district (Santa Cruz, Bolivia).} {\em Ecología en Bolivia, 45(2): 101-115}.

{1.} {Villarroel D.,} {\bf{J.N. Pinto-Ledezma}}, { T. Ruiz de Centurión and A. Parada. ({{\bf 2009}}) Relationship between the woody cover and herbs richness in three Cerrado {\em sensu lato} physiognomies (Cerro Mutún, Santa Cruz, Bolivia).} {\em Ecología en Bolivia, 44(2): 83-98}. 

{\bf BOOK CHAPTERS:}

{4.} {\bf{Pinto-Ledezma, J.N.}} {and J. Cavender-Bares. ({{\bf 2020}}). Using remote sensing for modeling and monitoring species distributions. In Cavender-Bares, J., J. Gamon and P. Townsend (Eds.)} {\em Remote Sensing of Plant Biodiversity. Springer Remote Sensing/Photogrammetry Series}.

{3.} {Cavender-Bares, J., A. Schweiger,} {\bf{J.N. Pinto-Ledezma}} {and J.E. Meireles. ({{\bf 2020}}). Applying remote sensing to biodiversity science. In Cavender-Bares, J., J. Gamon and P. Townsend (Eds.)} {\em Remote Sensing of Plant Biodiversity. Springer Remote Sensing/Photogrammetry Series}. 

{2.} {Villalobos, F.,} {\bf{J.N. Pinto-Ledezma}} {and J.A.F. Diniz-Filho. ({{\bf 2020}}). Evolutionary macroecology and the geographical patterns of Neotropical diversification. In Rull, V. and A.C. Carnaval (Eds.)} {\em Neotropical diversification: patterns and processes. Springer Nature AG}. 

{1.} {Contributing author in: Cavender-Bares, J. et al. Chapter 3 Status and trends of biodiversity and ecosystem functions underpinning nature’s benefit to people. In IPBES ({{\bf 2018}}): {The IPBES regional assessment report on biodiversity and ecosystem services for the Americas}. 207-362 Pp. Rice et al. (Eds).} {\em Secretariat of the Intergovernmental Science-Policy Platform on Biodiversity and Ecosystem Services, Bonn, Germany}. 

{\bf PAPERS IN OTHER PEER-REVIEWED JOURNALS:} 

{12.} {Moulatlet, G.M.,} {\bf{J.N. Pinto-Ledezma}}, {and F. Villalobos. (Accepted). Patrones globales de la distribución de angiospermas usando la diversidad beta.} {\em Ciencia Hoy.}

{11.} {\bf{Pinto-Ledezma, J.N.}}, {M.A. Montenegro and D. Villarroel. ({{\bf 2017}}). Historia Natural del Cerro Mutún V: la avifauna.} {\em Kempffiana, 13(2): 10-28}.

{10.} {Villarroel, D., G. Aramayo, M. Martínez, C. Proença, C. Munhoz, B. Klitgaard,} {\bf{J.N. Pinto-Ledezma}} {and M. Nee. ({{\bf 2017}}) Historia Natural del Cerro Mutún VI: flora y vegetación, checklist, estado de conservación y nuevos registros para Bolivia.} {\em Kempffiana, 13(2): 29-74}.

{9.} {*Pinto, M.A., *K. Mano-Cuellar, D. Villarroel and} {\bf{J.N. Pinto-Ledezma}}. {({{\bf 2017}}). Historia Natural del Cerro Mutún IV: la herpetofauna.} {\em Kempffiana, 13(1): 116-128}.

{8.} {\bf{Pinto-Ledezma, J.N.}} {and D. Villarroel. ({{\bf 2016}}). Historia Natural del Cerro Mutún I: síntesis geográfica, geofísica, climática y socioeconómica.} {\em Kempffiana, 12(2): 29-38}.

{7.} {*Pinto, M.A. and} {\bf{J.N. Pinto-Ledezma}}. {({{\bf 2015}}). Listado preliminar de anfibios de la propiedad Benevento (Santa Cruz, Bolivia).} {\em Kempffiana, 11(1): 23-27}.

{6.} {*Pinto M.A., D. García, K. Mano and} {\bf{J.N. Pinto-Ledezma}}. {({{\bf 2015}}). Listado de anfibios y reptiles de la propiedad Juan Deriba, Santa Cruz, Bolivia.} {\em Kempffiana, 11(1): 70-75}.

{5.} {*Mano K., *M.A. Pinto, *R. Sosa, D. Villarroel and} {\bf{J.N. Pinto-Ledezma}}. {({{\bf 2015}}). Reptile fauna of the Mutún region (Santa Cruz department, Bolivia): species list and conservation status.} {\em Kempffiana, 11(1): 66-69}.

{4.} {\bf{Pinto-Ledezma, J.N.}}, {T.J. Caballero, B. Flores, V.N. Perez, *K. Mano and *M.A. Pinto. ({{\bf 2014}}). Lista preliminar de las aves de la propiedad Juan Deriba, Santa Cruz, Bolivia.} {\em Kempffiana, 10(2): 1-11}.

{3.} {*Sosa R., L. Braga and} {\bf{J.N. Pinto-Ledezma}}. {({{\bf 2014}}). The amphibian fauna of the Southwest Amboró National Park, Santa Cruz, Bolivia.} {\em Kempffiana, 10(2): 31-35}.

{2.} {\bf{Pinto-Ledezma, J.N.}} {and M.A. Aponte. ({{\bf 2013}}) Algunas notas sobre la reproducción de aves en la Reserva de Vida Silvestre Ríos Blanco y Negro.} {\em Kempffiana, 9(1): 21-25}.

{1.} {\bf{Pinto-Ledezma, J.N.}}, {R. Sosa, M. Paredes, I. García, D. Villarroel and S. Muyucundo. ({{\bf 2011}}). The Hyacinth macaw ({\em Anodorhynchus hyacinthinus}): population status and its conservation in Bolivian Pantanal.} {\em Kempffiana, 7(2): 15-37.}. 

{\bf NON PEER-REVIEWED BOOK CHAPTERS:}

{4.} {Mostacedo B., M., Toledo, D. Villarroel,} {\bf{J.N. Pinto-Ledezma}}, G. Carreño-Rocabado, B. Flores and Y. Uslar. ({{\bf 2014}}). Memorias del IV Congreso Boliviano de Ecología. 4-6 de Junio 2014. {\em Universidad Autónoma Gabriel Rene Moreno, Santa Cruz, Bolivia.}

{3.} {Perotto-Baldivieso H.L., K. Riverro, } {\bf{J.N. Pinto-Ledezma}} {and A.B. Gill. ({{\bf 2012}}). Distributing biodiversity data through the web: The Geospatial Center for Biodiversity in Bolivia. 1252-1258 pp. In: Embrapa Informática Agropecuária/INPE. Anais 4º Simpósio de Geotecnologias no Pantanal.} {\em Instituto Nacional de Pesquisas Espaciais.} 

{2.} {Azurduy H. and} {\bf{J.N. Pinto-Ledezma}}. {({{\bf 2012}}). El escenario ecológico y geográfico. 6-13 pp. In: Azurduy and Rivero (Eds). Historia Natural de la Serranía Incahuasi.} {\em Museo de Historia Natural Noel Kempff Mercado and Total SE}.

{1.} {Villarroel D., L. Arroyo and} {\bf{J.N. Pinto-Ledezma}}. {({{\bf 2009}}). La vegetación de Bella Vista. 45-68 Pp. In: Arroyo and Churchill (Eds). Investigaciones botánicas en la región de Bella Vista, departamento de Santa Cruz, Bolivia: una base para la conservación.} {\em Museo de Historia Natural Noel Kempff Mercado and Missouri Botanical Garden}. 

\end{rSection}
%----------------------------------------------------------------------------------------
%	OTHER WRITING
%----------------------------------------------------------------------------------------

\begin{rSection}{Publications (in Review or Revision)} 

\normalfont 

{5.} {Guzmán, J.A.,} {\bf{J.N. Pinto-Ledezma}}, {D. Frantz, P.A Townsend, J. Juzwik, and J. Cavender-Bares. (First revision). Mapping oak wilt disease using phenological observations from space.} {\em Remote Sensing of Environment.} {Available online at:} \url {https://doi.org/10.1101/2023.05.25.542318}.

{4.} {Bala, A.,} {\bf{J.N. Pinto-Ledezma\dag}}, and {Z.A. Reshi\dag. (First revision). Phylogenetic relatedness of plant species co-occurring with an invasive alien plant species ({\em Anthemis cotula} L.) varies with elevation.} {\em Biological Invasions.} {Available online at:} \url {https://doi.org/10.1101/2023.03.10.532156}.

{3.} {Arango, A.,} {\bf{J.N. Pinto-Ledezma}}, { O. Soto-Rojas and F. Villalobos. (First revision). Evidences of widespread sympatry as the main driver of diversification for Emberizoidea (Aves).} {\em Proceedings of the Royal Society B.}

{2.} {Velasco, J.A., G. Campillo-García,} {\bf{J.N. Pinto-Ledezma}}, {and O. Villela-Flores. (First revision) Spatiotemporal dimensions of a reproductive life history trait in a spiny lizard radiation (Squamata: Phrynosomatidae).} {\em Evolution.} {Available online at:} \url {https://www.biorxiv.org/content/10.1101/2020.06.17.157891v1}.

{1.} {Souza, K.,} {\bf{J.N. Pinto-Ledezma}}, {R. Dobrovolski, M. Telles, T. Soares, C. Ruas and J.A.F. Diniz-Filho. (First revision). How to measure the influence of landscape population genetic structure: developing resistance surfaces using a pattern-oriented modeling approach.} {\em Genetica.} {Available online at:} \url {https://www.biorxiv.org/content/10.1101/2020.02.20.958637v1?rss=1}.

\end{rSection}

%\clearpage
%----------------------------------------------------------------------------------------
%	WRITING IN PROGRESS
%----------------------------------------------------------------------------------------

%\begin{rSection}{Intellectual Contributions in Preparation:}

%{4.} {\bf{Pinto-Ledezma, J.N.}}, {L. Kuczynski, J.A. Velasco, K. Marske, A. Carnaval, M. Papes, J. Cavender-Bares. Trait biogeography: the legacies of evolution and biogeographical origins.} {\em Target Journal: PNAS}.  

%{3.} {\bf{Pinto-Ledezma, J.N.}}, {J. Cavender-Bares +NutNet group. Evolutionary legacies on ecosystems: detecting phylogenetic responses of plants to global change.} {\em Target Journal: Science Advances}.  

%{2.} {\bf{Pinto-Ledezma, J.N.}}, {J.E. Meireles, F. Villalobos. Splendid isolation: diversification dynamics of the largest continental endemic vertebrate radiation.} {\em Target Journal: Evolution}. 

%----------------------------------------------------------------------------------------
%	TEACHING
%----------------------------------------------------------------------------------------

\begin{rSection}{Teaching and Advising}

\textbf{CORE TEACHING:}

\normalfont 

\begin{reSubsection}{University of Minnesota: }{Department of Ecology, Evolution \& Evolution }{}{
EEB 3534: \textbf{Biodiversity Science} (Instructor of record) \hfill Spring 2019-2023 \\ 
EEB 5534: \textbf{Biodiversity Science} (Instructor of record) \hfill Spring 2019-2023 \\ 
{Lab material at:
\url{https://jesusnpl.github.io/teaching/}} \smallskip

%EEB 3534/5534: \textbf{Biodiversity Science} (Instructor of record) \hfill Spring 2021 %\\ 
%EEB 5534: \textbf{Biodiversity Science} (Instructor of record) \hfill Spring 2021 \\ 
%{Lab material at:
%\url{https://jesusnpl.github.io/teaching/}} \smallskip

%EEB 3534/5534: \textbf{Biodiversity Science} (Instructor of record) \hfill Spring 2020 %\\ 
%EEB 5534: \textbf{Biodiversity Science} (Instructor of record) \hfill Spring 2020 \\ 
%{Lab material at:
%\url{https://github.com/jesusNPL/BiodiversityScience/Spring2020}} \smallskip 

%EEB 5534/5534: \textbf{Biodiversity Science} (Instructor of record) \hfill Spring 2019 %\\  
%EEB 3534: \textbf{Biodiversity Science} (Instructor of record) \hfill Spring 2019 \\ 
%{Lab material at:
%\url{https://github.com/jesusNPL/BiodiversityScience/Spring2019}} 
 
}

\end{reSubsection}

\begin{reSubsection}{Universidad Autónoma Gabriel René Moreno: }{Carrera de Biología }{}{
ZOO 344: \textbf{Vertebrate Zoology} (Guest Lecturer) \hfill Spring 2012, 2013 \\ \textbf{Landscape Ecology} (Guest Lecturer) \hfill Spring 2015, 2017 \\
{Master en Manejo de Recursos Naturales y Medio Ambiente} \smallskip 
} 
\end{reSubsection}

%\clearpage

\textbf{ADDITIONAL TEACHING:}

{\bf University of Minnesota: }{Biology Teaching \& Learning - NOL @ Itasca} \\
\textbf{Introduction to Programming Using R} (Instructor) \hfill July, 2023 \\
{Module material at:
\url{git@github.com:jesusNPL/NOL.git}}\smallskip 

{\bf University of Minnesota: }{Department of Ecology, Evolution \& Behavior } \\
\textbf{Evolutionary Ecology} (Guest Lecturer) \hfill May, 2023 \\
{Ecology (EEB 3407/5407) May Term 2023}\smallskip 

{\bf University of Minnesota: }{Department of Ecology, Evolution \& Behavior } \\
\textbf{NextGen species distribution models} (Lecturer) \hfill May, 2022 \\
{Lab material at:
\url{https://jesusnpl.github.io/NEON_Intro_NextGenSDM.html}}\smallskip 

{\bf University of Minnesota: }{Department of Fisheries, Wildlie and Conservation Biology } \\
\textbf{Introduction to patterns of biodiversity} (Guest Lecturer) \hfill Oct, 2019 \\
{Lab material at:
\url{https://github.com/jesusNPL/LargeScale}}\smallskip 

\begin{reSubsection}{Universidade Federal de Goiás: }{Department of Ecology }{}{
\textbf{Phylogenetic Comparative Methods} (Teaching Assistant) \hfill Spring 2016 \\ 
{Lab material at:
\url{http://dinizfilho.wixsite.com/dinizfilholab/}} \smallskip 
} 
\end{reSubsection}

{\bf Universidad Autónoma Gabriel René Moreno: }{Carrera de Biología } \\
ZOO 344: \textbf{Vertebrate Zoology} (Teaching Assistant) \hfill Spring 2003-2005, Fall 2003-2005 \\
{Six semesters}\smallskip 

{\bf Universidad Autónoma Gabriel René Moreno: }{Department of Botany } \\
\textbf{Introduction of statistics} (Instructor) \hfill Mar, 2012, 2013, 2014 \\
{Three intensive courses of one week each}\smallskip 

{\bf Universidad Autónoma Gabriel René Moreno: }{IV Congreso Boliviano de Ecología } \\
\textbf{Species distribution modeling with R} (Instructor) \hfill Jun, 2014 \\
{Three days course}\smallskip 

\textbf{DIRECTED STUDENT MENTORING:}

\textbf{PhD Thesis Committee:}

\textbf{Axel Arango García}, August 2019 - Present. External committee member, PhD Thesis/Project: `Effects of dispersal on the diversification of Emberizoidea (Aves, Passeriformes) in the New World'. Instituto de Ecología A.C., Xalapa, Mexico. 

\textbf{Felipe A. Toro Cardona}, February 2023 - Present. External committee member, PhD Thesis/Project: `Macroecological patterns and niche evolution of Hylids in America'. Universidad de Antioquia, Colombia. 

\textbf{Undergraduate Mentoring:}

\textbf{Zoe Karwowski} - EEB, University of Minnesota

\textbf{Past Graduate Advisees:}

\textbf{Marco Aurelio Pinto Viveros}, February 2017 – August 2019. Master Science Thesis: `The amphibians as a model of biological control in agricultural areas of Santa Cruz, Bolivia', Master program Manejo de Recursos Naturales y Medio Ambiente, Universidad Autónoma Gabriel René Moreno, Santa Cruz de la Sierra, Bolivia.

\textbf{Past Undergraduate Advisees:}

\textbf{Katherine Mano Cuellar}, March 2012 – July 2014. Distinction in All Subjects. {\em Magna Cum Laude}. Undergraduate Project. `Effects of land use change on amphibian community composition in central Santa Cruz, Bolivia'. Carrera de Biología, Universidad Autónoma Gabriel René Moreno, Santa Cruz de la Sierra, Bolivia.

\textbf{Marco Aurelio Pinto Viveros}, March 2012 – December 2014. Distinction in All Subjects. {\em Magna Cum Laude}. Undergraduate Project: `The herpetofauna of the Mutún region, Santa Cruz, Bolivia'. Carrera de Ciencias Ambientales, Universidad Autónoma Gabriel René Moreno, Santa Cruz de la Sierra, Bolivia.

\textbf{Ronald Sosa Escalante}, March 2013 – July 2016. Undergraduate Thesis: `Estudio de la mortalidad de serpientes atropelladas en la carretera Antigua a Cochabamba, Provincia Florida, Santa Cruz, Bolivia, Carrera de Biología, Universidad Autónoma Gabriel René Moreno, Santa Cruz de la Sierra, Bolivia.

\end{rSection}

%\clearpage

%----------------------------------------------------------------------------------------
%	INVITED TALKS
%----------------------------------------------------------------------------------------
%\clearpage

\begin{rSection}{Presentations} 
\normalfont 

\em{*Last five years}

\begin{sSubsection}{Invited Presentation - Plant diversity across dimensions}{ \hfill }{August, 2023}{ Research at Multiple Scales: A Vision for Continental Scale Biology - The National Academies of Sciences, Engineering, and Medicine }{Webinar}
\end{sSubsection}

\begin{sSubsection}{Contributed Presentation - No branch left behind: tracking terrestrial biodiversity from a phylogenetic completeness perspective}{ \hfill }{August, 2023}{ Ecological Society of America Annual Meeting }{Portland, OR}
\end{sSubsection}

\begin{sSubsection}{Contributed Presentation - How spectral biology and remote sensing can inform biodiversity management and conservation}{ (Co-author) }{July, 2023}{ Botany Annual Conference }{Boise, ID}
\end{sSubsection}

\begin{sSubsection}{Invited Presentation - Trait biogeography: the role of past distributions}{ \hfill }{Apr, 2023}{ Symposium on the Biogeography of Behavior, University of Oklahoma }{Norman, OK}
\end{sSubsection}

\begin{sSubsection}{Invited Presentation - Harnessing Bird and Remote Sensing Data to Understand Biodiversity Change Over Space and Time}{ \hfill }{Jan, 2023}{ Saint Paul Audubon Society }{Saint Paul, MN}
\end{sSubsection}

\begin{sSubsection}{Contributed Presentation - Mapping phylogenetic composition and diversity in temperate forests for conservation and disease detection}{ (Co-author) }{Dec, 2022}{ AGU Annual Meeting }{Chicago, IL}
\end{sSubsection}

\begin{sSubsection}{Invited Presentation - Can we use remote sensing products to predict and monitor biodiversity?}{ \hfill }{Aug, 2022}{ Panorama Actual de las Ciencias Atmosféricas y del Cambio Climático 2022 }{Mexico DF, Mexico}
\end{sSubsection}

\begin{sSubsection}{Contributed Presentation - Assessing the link between spectral diversity and functional diversity}{ \hfill }{Jul, 2022}{ Annual Meeting of the Association for Tropical Biology and Conservation }{Cartagena, Colombia}
\end{sSubsection}

\begin{sSubsection}{Contributed Presentation - The role evolutionary and biogeographic imprints on plant-trait distributions and ecosystems functioning}{ \hfill }{Jun, 2022}{ World Biodiversity Forum }{Davos, Switzerland}
\end{sSubsection}

\begin{sSubsection}{Invited Presentation - Macroevolutionary and ecological processes as drivers of species co-occurrences}{ \hfill }{Feb, 2022}{ Ecology, Evolution and Environmental Biology Seminar Series, Columbia University}{New York, NY}
\end{sSubsection}

\begin{sSubsection}{Contributed Presentation - Plant diversity across dimensions: coupling biodiversity measures from the ground and the sky}{ \hfill }{Dec, 2021}{ AGU Annual Meeting }{New Orleans, LA}
\end{sSubsection}

\begin{sSubsection}{Contributed Poster - Modeling species distributions using remote sensing data: the Eastern temperate forests of the US as a case study}{ }{Dec, 2021}{ AGU Annual Meeting }{New Orleans, LA}
\end{sSubsection}

\begin{sSubsection}{Contributed Presentation - Plant hydraulics and rainfall niches: a mechanistic approach to explain species distributions across tropical biomes}{ (Co-author) }{Dec, 2021}{ AGU Annual Meeting }{New Orleans, LA}
\end{sSubsection}

\begin{sSubsection}{Contributed Presentation - Detection of oak wilt disease from tree to landscape Scales}{ (Co-author) }{Dec, 2021}{ AGU Annual Meeting }{New Orleans, LA}
\end{sSubsection}

\begin{sSubsection}{Invited Presentation - Predictive ecological and evolutionary science}{ }{Sep, 2021}{ School of Natural and Environmental Sciences, Newcastle University (Online)}{Newcastle, UK}
\end{sSubsection}

\begin{sSubsection}{Invited Presentation - Plant community structure and detection}{ }{Jun, 2021}{ Institute of Biology, Leipzig University (Online)}{Leipzig, Germany}
\end{sSubsection}

\begin{sSubsection}{Invited Presentation - Plant biodiversity: community structure, composition and detection}{ }{Jan, 2021}{ Ecology, Evolution and Behavior Seminar Series, University of Minnesota (Online)}{St Paul, MN}
\end{sSubsection}

\begin{sSubsection}{Contributed Poster - Integrating biodiversity observations with airborne and satellite data to predict shifts in assemblage diversity and composition underg global change}{ }{Jan, 2021}{NSF Macrosystems Biology and NEON Enabled Science PI Meeting}{Online meeting}
\end{sSubsection}

\begin{sSubsection}{Invited Presentation - Macroecology and macroevolution in the Neotropics}{ }{Nov, 2020}{Department of Geography, Federal University of Rio Grande do Norte (Online)}{RGN, Brazil}
\end{sSubsection}

\begin{sSubsection}{Invited Presentation - Introduction to graphical models}{ }{Jan, 2020}{Evoutionary Biology Network, Institute of Ecology}{Xalapa, Mexico}
\end{sSubsection}

\begin{sSubsection}{Contributed Presentation - The role of ecology and evolution on the assembly and species co-occurrence at different spatial and temporal scales}{ }{Sep, 2019}{Grand Challenges in Biology Symposium, University of Minnesota}{St Paul, MN}
\end{sSubsection}

\begin{sSubsection}{Contributed Poster - Testing Darwin's naturalization conundrum based on taxonomic, phylogenetic and functional dimensions of vascular plant diversity}{ }{Aug, 2019}{Ecological Society of America Annual Meeting}{Louisville, KY}
\end{sSubsection}

\begin{sSubsection}{Invited Presentation - Wildlife management and indigenous people in Bolivian lowlands}{ }{Apr, 2019}{Mano a Mano International Partners}{St Paul, MN}
\end{sSubsection}

\begin{sSubsection}{Contributed Presentation - Integrated Global Biodiversity Detection: Plant Spectra, Phylogenetics, and Enhanced Species Distribution Models}{ (Co-author) }{Dec, 2018}{AGU Annual Meeting}{Washington DC}
\end{sSubsection}

\begin{sSubsection}{Invited Presentation - Evolutionary macroecology}{ }{Dec, 2018}{Museo de Historia Natural Noel Kempff Mercado}{Santa Cruz, Bolivia}
\end{sSubsection}


\end{rSection}

%\clearpage
%----------------------------------------------------------------------------------------
%	ORGANIZATION
%----------------------------------------------------------------------------------------
\begin{rSection}{Symposia and Workshops Organized}

\normalfont

\begin{sSubsection}{Round table on Diversity in Biodiversity Science}{ (Organization committee) }{Oct, 2020}{Biodiversity Research Coordination Network (RCN)}{Online meeting}
\end{sSubsection}

\begin{sSubsection}{X Bolivian Congress of Ornithology }{ (Scientific committee) }{Oct, 2019}{Asociación Boliviana de Ornitología and Universidad San Francisco Xavier}{ Sucre, Bolivia}
\end{sSubsection}

\begin{sSubsection}{IV Bolivian Congress of Ecology }{ (Vice-president and Scientific committee) }{Jun, 2014}{Asociación Boliviana de Ecología }{ Santa Cruz, Bolivia}
\end{sSubsection}

\begin{sSubsection}{Climate Change and Water Use }{ (Organization committee) }{Oct, 2010}{ADAPCLIM conference }{ Asunción, Paraguay}
\end{sSubsection}

\begin{sSubsection}{First Encounter on Knowledge and Management of the Pantanan and Chiquitania in the context of the Paraguay River Basin}{ (Organization committee) }{Jun, 2010}{Museo Noel Kempff Mercado/SINERGIA }{ Puerto Quijarro, Bolivia}
\end{sSubsection}

\end{rSection}

%----------------------------------------------------------------------------------------
%	SERVICE
%----------------------------------------------------------------------------------------
\begin{rSection}{Service and Outreach Activities}

\textbf{SERVICE AS EDITOR:} %\smallskip \\ 
\normalfont

\item 1. Subject-Matter Editor | {\bf Ecological Monographs} (The Ecological Society of America) 
\item 2. Editorial Board | {\bf Kempffiana} (Museo de Historia Natural Noel Kempff Mercado) 
\item 3. Subject-Matter Editor - special issue: {\em Applications of spectral biology and remote sensing for deciphering the causes and consequences of plant biodiversity across scales} | {\bf Ecology} (The Ecological Society of America)
\item 4. Guest Associate Editor in Models in Ecology and Evolution | {\bf Frontiers in Ecology and Evolution}

\textbf{SERVICE AS REVIEWER:} \smallskip \\ 
\textbf{Web of Science ResearcherID: E-7984-2014}

\normalfont

I served as a reviewer panelist for the {\bf US National Science Foundation} in February 2022. In 2020, I served as an expert reviewer for the 2020 Red List of birds (eastern South America). I am also a Spanish-language reviewer for {\bf The American Naturalist}. The aim is helping The American Naturalist to expand the communications reach of the world-class science that nonnative English speakers produce. 

I have served as a reviewer in areas of macroecology and macroevolution, biogeography, theoretical ecology, community ecology, and ecological modeling in journals, including: \smallskip 
\begin{multicols}{2}

Nature \\
Ecology(7) \\
Frontiers in Ecology and the Environment \\
Ecological Monographs(2) \\ 
Philosophical Transactions of the Royal Society B \\
Ecology Letters \\
Journal of Biogeography(12) \\ 
Global Ecology and Biogeography(11) \\
Diversity and Distributions(3) \\ 
Journal of Aninal Ecology(2) \\
Journal of Ecology(4) \\ 
Methods in Ecology and Evolution(2) \\ 
Ecology and Evolution(2) \\ 
Journal of Field Ornithology(2) \\
EMU (Australian Journal of Ornithology)(2) \\ 
Kempffiana \\ 
Oecología Australis \\
El Hornero (Ornitología Neotropical) \\ 
Communications Biology \\
Biological Conservation \smallskip

\columnbreak

PNAS(3) \\
The American Naturalist(2) \\ 
Systematic Biology(2) \\
Nature Communications(3) \\
New Phytologist(7) \\ 
Molecular Biology and Evolution \\ 
Biological Journal of the Linnean Society(2) \\
Remote Sensing in Ecology and Conservation(3) \\ 
Biotropica(3) \\ 
Annals of Botany \\ 
Journal of Vegetation Science(2) \\ 
Journal of Plant Ecology \\ 
Prespectives in Ecology and Conservation(2) \\
Journal of Zoolog. Syst. Evol. Research \\
Biodiversity and Conservation(2) \\
Forest Ecology and Management(2) \\ 
NPJ Biodiversity(2) \\
Plos One \\
PeerJ \smallskip

\end{multicols}

\textbf{NATIONAL AND INTERNATIONAL SERVICE ACTIVITIES:} \smallskip 

\item 1. \textbf{Official Bolivian member} for the Society of Wetlands Scientists, South American Chapter. 

\item 2. \textbf{Research Advisor in Ecology and Natural Resources}: National Academy of Sciences of Bolivia, Santa Cruz Chapter ({\em Scientia Crucensis}). 

\textbf{OUTREACH:} \smallskip \\ 
While working at the Noel Kempff Mercado Natural History Museum in Santa Cruz de la Sierra, I participated in outreach activities with visitors. I participated in guided visits from elementary and high school students and to the ornithological collection at the Museum, where we explored the role of scientific collections in science and society and how we learn about and document biodiversity.

As a postdoctoral fellow at the university of Minnesota, I have been participating in different science outreach programs. One of the programs involves bringing elementary school students (usually 5th and 6th graders) to the University of Minnesota. I lead part of the biodiversity sessions, in which the students can see and manipulate different plant species in the greenhouse and learn about the role of environmental conditions in species diversity, function and adaptations. I have also participated in the Market Science, which aims to connect people with science. In Market Science we created hand-one science activities for children in Farmer’s markets. 
%For example, my last experience in Market Science (September 2019), Laura Toro (PhD candidate from Colombia) and I created activities related to the impacts of wildfire and human induced fires in South America. 

More recently, I have co-organized an online round-table entitled {\bf Diversifying Biodiversity Science} under the umbrella of the RCN: Cross-Scale Processes Impacting Biodiversity collaborative project, in which several panelist were invited to talk about their experiences regarding the inequalities in biodiversity science and how they are working to make our work more inclusive. Link to the round-table video on YouTube \url{https://www.youtube.com/watch?v=CYlt3bMIsEs&t=296s} 

Moreover, I gave a public talk at “Mano a Mano” International Partners (April 2019, \url{https://manoamano.org}) on wildlife management and indigenous people in Bolivian lowlands. In this talk, I highlighted the importance of working with local communities in order to preserve the natural capital. 


\end{rSection}

%----------------------------------------------------------------------------------------
%	SKILLS
%----------------------------------------------------------------------------------------

\begin{rSection}{Skills}

\normalfont

\begin{tabular}{ @{} >{\bfseries}l @{\hspace{6ex}} l }
Languages & First language: Spanish (speaking, reading, and advanced writing) \\ 
 & Proficient: English, Portuguese (speaking, reading, and writting) \\ 
 & Familiar with: Guaraní, Quechua \smallskip \\

Computer Programming & Advanced: \textsc{R}, \textsc{Markdown} \\
 & Intermediate: \LaTeX, \textsc{Stan}, \textsc{Git}, \textsc{RevBayes}, \textsc{Matlab}, \textsc{ArcGis} \\ 
 & Familiar with: \textsc{Julia}, \textsc{Python}, \textsc{C++}, \textsc{Bash}, \textsc{Envi}

\end{tabular}

%Open source contributions: \\
%Tidy labs for Introduction to Statistical Learning: \url{https://github.com/SmithCollege-SDS/tidy-islr} \\ 
%Labs accompanying OpenIntro Introduction to Statistics with Randomization and Simulation \url{https://github.com/beanumber/oiLabs-mosaic} \\ 
%R packages: 
%Many more projects hosted on GitHub, \url{https://github.com/AmeliaMN} \\
%Additional projects contributed to: 
\end{rSection}

%\clearpage

%----------------------------------------------------------------------------------------
%	EXPERIENCE
%----------------------------------------------------------------------------------------
\begin{rSection}{Experience with Scientific Collections and Curation}

\normalfont

\textbf{Museo de Historia Natural Noel Kempff Mercado, Santa Cruz de la Sierra, Bolivia} \smallskip 
\item 1.	\textbf{Assistant curator}: identification, preparation and maintenance of bird specimens collected in the field (2009 – 2014). 
\item 2.	\textbf{Coordinator}: development of the Geospatial Centre for Biodiversity in Bolivia—for the vertebrate collection at the Museum in collaboration with a team of taxonomists. \url{http://www.museonoelkempff.org/cgb/}. (2011 – 2013). 
\item 3.	\textbf{Field coordinator of biological inventories} in National Protected Areas of Bolivia, including Ríos Blanco y Negro Wildlife Reserve (2009), Otuquis National Park (2011, 2013).

\end{rSection}

%\clearpage
%----------------------------------------------------------------------------------------
%	MEMBERSHIPS
%----------------------------------------------------------------------------------------

\begin{rSection}{Memberships in Professional Societies}

\normalfont

\begin{esSubsection}{American Association for the Advancement of Science }{(AAAS)}{2020-present}{}{}
\end{esSubsection}

\begin{esSubsection}{Sigma Xi, The Scientific Research Honor Society }{(\textSigma  \textXi)}{2021-present}{}{}
\end{esSubsection}

\begin{esSubsection}{American Geophysical Union }{(AGU)}{2019-present}{}{}
\end{esSubsection}

\begin{esSubsection}{Ecological Society of America }{(ESA)}{2014-present}{}{}
\end{esSubsection}

\begin{esSubsection}{National Academy of Sciences of Bolivia, Santa Cruz Chapter }{}{2013-Present}{}{}
\end{esSubsection}


\begin{esSubsection}{Ornithological Society of Bolivia }{(ASBOR)}{2012-Present}{}{}
\end{esSubsection}

\begin{esSubsection}{Ecological Society of Argentina }{(ASAE)}{2006-Present}{}{}
\end{esSubsection}

\begin{esSubsection}{Ecological Society of Bolivia }{(ABECOL)}{2006-Present}{}{}
\end{esSubsection}

\begin{esSubsection}{Society of Wetlands Scientists, South American Chapter }{(SWS)}{2010-2014}{}{}
\end{esSubsection}

\begin{esSubsection}{International Society for Salt Lake Research }{(ISSLR)}{2010-2014}{}{}
\end{esSubsection}

\begin{esSubsection}{Community of Wildlife Management in Latin America }{(COMFAUNA)}{2009-Present}{}{}
\end{esSubsection}

\begin{esSubsection}{Society of Conservation Biology, Bolivian Chapter }{(SCB)}{2007-2012}{}{}
\end{esSubsection}

\end{rSection}



%----------------------------------------------------------------------------------------
%	References
%----------------------------------------------------------------------------------------
\clearpage
%\input{references.tex} %Comment out to get this to compile




\end{document}
